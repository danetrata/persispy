\documentclass{beamer}

\mode<presentation>{\usetheme{poster}}

% Load packages here
\usepackage{amsmath,amssymb,array}
\usepackage{graphicx,tcolorbox}
\usepackage{ragged2e}
\tcbuselibrary{listings,breakable,most,hooks,skins}
\graphicspath{{./graphics/}}
\usepackage[orientation=landscape,size=custom,width=48,height=36,scale=.6,debug]{beamerposter}
\usepackage{mathptmx}
\usepackage{verbatim}
\renewcommand\sfdefault{ptm}
\def\Z{\mathbb Z}
\def\N{\mathbb N}
\def\Q{\mathbb Q}
\def\R{\mathbb R}
\def\C{\mathbb C}
\def\F{\mathbb F}
\def\P{\mathbb P}
\def\E{\mathbb E}
\def\G{\mathbb G}
% Header and footer 
\newcommand{\footleft}{http://mcl.math.uic.edu/}
\newcommand{\footright}{}
\title{Vietoris-Rips complexes, random points on varieties, and persistent homology}
\author{Daniel Etrata \quad Mason Boeman}
\institute{Mentors: Benjamin Antieau and J\={a}nis Lazovskis}

% Main document
\begin{document}
\begin{frame}{}
\begin{columns}[t]

%-- Column 1 ---------------------------------------------------
\begin{column}{0.32\linewidth}

%-- Block 1-1
\begin{block}{Summary}

The project takes a variety of n-dimensions intersected by random lines to generate points. From this point cloud, an undirected neighborhood graph is constructed, where then the edges are assigned a weight. Then, the neighborhood graph undergoes the Vietoris-Rips expansion by constructing cofaces for which the vertex is maximal.
\end{block}



%-- Block 1-2
\begin{block}{Motivation}
The goal of this project is to use persistent homology to study the topology of random complex algebraic varieties, geometric figures described as the solution sets of system of polynomial equations. 

\end{block}

%-- Block 1-3
\begin{block}{Using PHCpack to find points}
PHCpack is a polynomial system solver using homotopy continuation methods. Let $f(x) = 0$ be the target system. PHC constructs $g(x)=0$ start system. Then let $\gamma$ be the accessibility constant, $\gamma \in \C$ and $t$ be the continuation parameter, $t \in [0,1]$. The homotopy $h$ is defined as
\begin{equation*}
h(x,t) = \gamma \cdot (1-t)\cdot g(x) + t \cdot f(x) = 0
\end{equation*}
\begin{equation*}
h(x, 0) =g(x)
\end{equation*}
\begin{equation*}
h(x, 1) = f(x)
\end{equation*}
\includegraphics[width=.5\columnwidth]{sphere}
\includegraphics[width=.5\columnwidth]{plot2d_5}


\end{block}


\end{column}%1

%-- Column 2 ---------------------------------------------------
\begin{column}{0.32\linewidth}

%-- Block 2-1
\begin{block}{Neighborhood graphs}

The neighborhood graph of a set $S$ with a parameter $e$ is the graph with vertices in $S$, where en edge is contained in the neighborhood graph if and only if the distance between it's endpoints is less than $e$. Our quick edge algorithm takes as input a set of points $S$ of order $n$ in affine space, and some real number $e$, and produces a list of all edges that have length less than $e$. Our method uses recursion. If our ambient space has dimension $d$, let $k$ be some positive integer less than $d$. 
\newline
\newline
First we compute the median value of the $k$-th coordinate of all the points in $S$, and call this $m$. Now we construct four subsets of $S$: $A$, $B$, $A^*$, and $B^*$. $A$ is the set of points whose $k$-th coordinate is less than $m$. $B$ is defined as $S-A$. $A^*$ and $B^*$ are subsets of $A$ and $B$ respectively where points in these sets have kth coordinate within $e$ of $m$. We add an edge to our list for every pair $(a,b)$ where $a$ in $A^*$ and $b$ in $B^*$, and they are within $e$ of each other. We then call the algorithm recursively, replacing $S$ with $A$ and $B$, each half the size.

\includegraphics[width=1\columnwidth]{plot2d_ng_7}
\end{block}

\end{column}%2

%-- Column 3 ---------------------------------------------------
\begin{column}{0.32\linewidth}

%-- Block 3-1
\begin{block}{Persistent homology}

\end{block}

%-- Block 3-2
\begin{block}{Experiments}

\end{block}

%-- Block 3-3
\begin{block}{Conclusion}

\end{block}

\end{column}%3

\end{columns}
\end{frame}
\end{document}
